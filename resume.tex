\documentclass[11pt, letterpaper]{article}

\usepackage{multicol}
\usepackage{blindtext}
\usepackage{hyperref}
\usepackage{titlesec}
\usepackage{titling}
\usepackage{enumitem}
\usepackage{amsmath}
\setlist[itemize]{noitemsep,topsep=0pt,itemindent=0em}

\usepackage{geometry}
\geometry{left=1in, top=1in, right=1in, bottom=1in, footskip=.5cm}

\usepackage{xspace}
\newcommand{\vbar}{\textbar\xspace}

\leftskip=\the\parindent
\setlength{\parindent}{0pt}

\titleformat{\section}
  {\normalfont\large\bfseries}
  {}
  {0pt}
  {}
  [{\titlerule[0.8pt]}]

\titlespacing{\section}
  {0pt}
  {1em}
  {\baselineskip}

\newcommand{\comment}[1]{}

\pagestyle{empty}

\title{1295 Military Trail, Toronto, ON, M1C 1A4}
\author{Benjamin Chislett}
\date{}

\begin{document}

\begin{center}
  \Huge\theauthor
  \par
  \large\thetitle
  \par
  \large{benjamin.chislett@mail.utoronto.ca}
  \par
  \large\url{https://github.com/benchislett}
\end{center}

\begin{section}{Profile}

First year Computer Science student with a thirst for knowledge and a passion for learning.
Skilled programmer with experience in full-stack development, research, and machine learning.
\\

\textbf{Key Competencies:} AWS \vbar node.js \vbar npm \vbar Python \vbar C/C++ \vbar GPU Computing \vbar Agile \vbar Problem-Solving \vbar Work ethic \vbar Teamwork \vbar Communication \vbar Adaptability \vbar Organization

\end{section}

\begin{section}{Experience}

\textbf{Research Intern}
\hfill
\textit{Jul 2019 - Sep 2019}\\*
\textit{Okinawa Institute of Science and Technology, Okinawa Prefecture, Japan}

\begin{itemize}
  \item Maintained GPUE: a CUDA/C++ application for simulating Bose-Einstein condensates
  \item Authored GPUE.jl: a Julia-GPU implementation of GPUE\\
\end{itemize}

\textbf{Software Developer}
\hfill
\textit{Sep 2018 - Jul 2019}\\*
\textit{Mysa Smart Thermostats, St. John's, NL}
\begin{itemize}
  \item Maintained a full stack javascript application for browsing and updating an AWS backend
  \item Developed a query library for interacting with AWS through object-oriented abstractions
\end{itemize}

\end{section}

\begin{section}{Projects}
\textbf{Convolutional Autoencoders for Image Compression}
\begin{itemize}
  \item Researched machine-learning and computer vision techniques
  \item Developed a novel image compression algorithm using convolutional autoencoders and data compression\\
\end{itemize}

\textbf{Evolutionary Algorithms for Polygonal Approximation of Images}
\begin{itemize}
  \item Implemented meta-heuristic optimization of image approximation with hill-climbing genetic algorithms
  \item Used low-level image rendering APIs to create live video output over time\\
\end{itemize}

\textbf{Time Management Application Integrated With Trello}
\begin{itemize}
  \item Developed Pomorello, an application for time management and efficiency analytics, available natively in Trello
  \item Implemented a self-contained backend and simple analytics engine entirely within Trello through the iFrame callback API
\end{itemize}

\end{section}

\begin{section}{Education}
\textbf{Honours Bachelor of Science, Computer Science}
\hfill
\textit{Sep 2019 - Present}\\*
\textit{University of Toronto Scarborough, Scarborough, ON}

\end{section}

\end{document}

