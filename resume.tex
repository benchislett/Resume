\documentclass[11pt, letterpaper]{article}

\usepackage{multicol}
\usepackage{blindtext}
\usepackage{hyperref}
\usepackage{titlesec}
\usepackage{titling}

\usepackage{geometry}
\geometry{left=2.0cm, top=1.5cm, right=2.0cm, bottom=2.0cm, footskip=.5cm}

\leftskip=\the\parindent
\setlength{\parindent}{0pt}

\titleformat{\section}
  {\normalfont\large\bfseries}
  {}
  {0pt}
  {}
  [{\titlerule[0.8pt]}]

\titlespacing{\section}
  {0pt}
  {1em}
  {\baselineskip}

\newcommand{\comment}[1]{}

\title{Computer Scientist}
\author{Benjamin Chislett}
\date{}

\begin{document}

\begin{center}
  \Huge\theauthor
  \par
  \huge\thetitle
  \par
  \large\url{https://github.com/benchislett}
\end{center}

\begin{section}{Skills}

\begin{multicols}{2}
\textbf{Machine Learning}
  \textit{
  \begin{itemize}
  \item Variational Autoencoders
  \item Convolutional Neural Networks
  \item Recurrent Neural Networks
  \item Generative Adversarial Networks
  \item Evolutionary Algorithms
  \end{itemize}
  }

\textbf{Frameworks}
  \textit{
  \begin{itemize}
  \item PyTorch
  \item Tensorflow
  \item React
  \item CUDA
  \item AWS-SDK
  \end{itemize}
  }
\end{multicols}

\end{section}

\begin{section}{Experience}

\textbf{Okinawa Institute of Science and Technology - Research Intern}
\hfill
\textit{July 2019 - September 2019}

Maintained GPUE: a CUDA/C++ application for simulating Bose-Einstein condensates\\
Authored GPUE.jl: a Julia-GPU implementation of GPUE\\

\textbf{Mysa Smart Thermostats - Software Developer}
\hfill
\textit{September 2018 - July 2019}

Maintained a full stack javascript application for browsing and updating an AWS backend\\
Developed a query language for interacting with AWS through object-oriented abstractions
\end{section}

\begin{section}{Projects}
\textbf{Convolutional Autoencoders for Image Compression}\\*
Developed a novel image compression algorithm through convolutional variational autoencoders\\*

\textbf{Evolutionary Algorithms for Polygonal Approximation of Images}\\*
Implemented hill-climbing and a genetic algorithm for metaheuristic optimization of image approximation with a fixed number of semi-transparent polygons\\

\end{section}

\begin{section}{Education}
\textbf{University of Toronto Scarborough}
\hfill
\textit{2019-Present}

First year student enrolled in Honours Computer Science Co-op, full-time\\

\textbf{Memorial University of Newfoundland}
\hfill
\textit{2018-19}

Part-time student through the Concurrent Studies program, 4.0 GPA over 3 courses\\
\end{section}

\end{document}

