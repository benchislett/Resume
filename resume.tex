\documentclass[11pt, letterpaper]{article}

\usepackage{multicol}
\usepackage{hyperref}
\usepackage{titlesec}
\usepackage{titling}
\usepackage{enumitem}
\usepackage{amsmath}
\usepackage{fontawesome}
\setlist[itemize]{noitemsep,topsep=0pt,itemindent=0em}

\usepackage{geometry}
\geometry{left=2cm, top=2cm, right=2cm, bottom=2cm, footskip=.5cm}

\usepackage{xspace}
\newcommand{\vbar}{\textbar\xspace}

\leftskip=\the\parindent
\setlength{\parindent}{0pt}

\titleformat{\section}
  {\normalfont\large\bfseries}
  {}
  {0pt}
  {}
  [{\titlerule[0.8pt]}]

\titlespacing{\section}
  {0pt}
  {1em}
  {\baselineskip}

\newcommand{\comment}[1]{}

\pagestyle{empty}

\title{}
\author{Benjamin Chislett}
\date{}

\begin{document}

\begin{center}
  \Huge\theauthor
  \par
  \vspace{1mm}
  \large{chislett.ben@gmail.com} $|$ \large{benjamin.chislett@epfl.ch}
  \par

\end{center}

\begin{section}{Experience}

\textbf{Research Intern}
\hfill
\textit{May 2022 - Jan 2023}\\*
\textit{Dynamic Graphics Project, University of Toronto}
\begin{itemize}
  \item Researched intrinsic triangulations and coarsening applications for multigrid methods on 3D surfaces
  \item Developed a novel high-performance intrinsic mesh decimation library in C++ and Python
  \item Co-authored "Surface Simplification using Intrinsic Error Metrics" in ACM Transactions on Graphics Journal, featured at SIGGRAPH 2023 \\
\end{itemize}

\textbf{Research Intern}
\hfill
\textit{May 2021 - Sep 2021}\\*
\textit{EcoSystem Research Lab, University of Toronto}
\begin{itemize}
  \item Researched auto-scheduling for machine learning compilers in TVM and low-level optimizations for out-performing cuBLAS in CUDA workloads
  \item Implemented and evaluated novel auto-scheduling algorithms based on the AutoTVM and Ansor
  \item Received NSERC Undergraduate Student Research Award (USRA) funding \\
\end{itemize}

\textbf{Machine Learning Engineer}
\hfill
\textit{Apr 2021 - Aug 2021, July - Aug 2023}\\*
\textit{Activeloop AI, California (Remote)}
\begin{itemize}
  \item Designed infrastructure for a cloud machine learning data platform in Python
  \item Developed machine learning solutions using PyTorch, Tensorflow, and AWS cloud computing
  \item Architected distributed tensor database systems in C++ and Python, designed and implemented database abstractions and core functionality \\
\end{itemize}

\textbf{Software Developer}
\hfill
\textit{Sep 2018 - Jul 2019, May 2020 - Dec 2020}\\*
\textit{Mysa Smart Thermostats, St. John's, NL}
\begin{itemize}
  \item Developed and maintained a full-stack TypeScript web interface for an AWS backend
  \item Authored a suite of libraries for interacting with AWS at multiple layers of abstraction
  \item Developed various new features for a React-Native mobile application
  \item Architected a data pipeline used to create a data lake and perform analytics using IaC and SQL \\
\end{itemize}

\textbf{Research Intern}
\hfill
\textit{Jul 2019 - Sep 2019}\\*
\textit{Okinawa Institute of Science and Technology, Okinawa Prefecture, Japan}
\begin{itemize}
  \item Researched the Compressive Split-Step Fourier Method for solving Gross-Pitaevskii systems.
  Performed mathematical modeling and validation, optimized numerical GPU solvers, and developed theoretical foundations for order-of-magnitude speedups using compressed frequency methods.
  \item Maintained GPUE: a CUDA/C++ application for simulating Quantum effects of superfluids
  \item Authored GPUE.jl: a high performance JuliaLang-GPU implementation of GPUE
\end{itemize}

\end{section}

\pagebreak

\begin{section}{Education}

\textbf{First-Year Doctoral Fellow, Computer Science}
\hfill
\textit{Sep 2023 - Present}\\*
\textit{EPFL, Lausanne, Vaud, Switzerland}
\begin{itemize}
  \item Member of Wenzel Jakob's Realistic Graphics Lab
  \item Teaching Assistant for Advanced Computer Graphics (CS440), Spring 2024 Semester
  \item Actively developing high utilization hardware-accelerated machine learning primitives for differentiable rendering in PTX/CUDA \\
\end{itemize}

\textbf{Honours Bachelor of Science, Computer Science}
\hfill
\textit{Sep 2019 - May 2023 (05/23)}\\*
\textit{University of Toronto, Scarborough, ON}
\begin{itemize}
  \item Cumulative GPA: 3.96
  \item ICPC North America Finalist, 2020/21 \\
\end{itemize}

\textbf{Teaching Assistant at University of Toronto}
\begin{itemize}
  \item Artificial Intelligence (CSCD84), \\*Principles of Programming Languages (CSCC24) \hfill \textit{Winter 2022/23}
  \item Computability and Computational Complexity (CSCC63), \\*Algorithm Design and Analysis (CSCC73), Computer Graphics (CSCD18) \hfill \textit{Fall 2022}
  \item Introduction to the Theory of Computation (CSCB36), Linear Algebra 2 (MATB24) \hfill \textit{Summer 2022}
  \item Introduction to the Theory of Computation (CSCB36) \hfill \textit{Fall 2021}
  \item Linear Algebra 1 (MATA22), Introduction to Computer Science 2 (CSCA48) \hfill \textit{Winter 2020/21}
\end{itemize}

\end{section}

\end{document}

