\documentclass[11pt, letterpaper]{article}

\usepackage{multicol}
\usepackage{hyperref}
\usepackage{titlesec}
\usepackage{titling}
\usepackage{enumitem}
\usepackage{amsmath}
\setlist[itemize]{noitemsep,topsep=0pt,itemindent=0em}

\usepackage{geometry}
\geometry{left=2cm, top=2cm, right=2cm, bottom=2cm, footskip=.5cm}

\usepackage{xspace}
\newcommand{\vbar}{\textbar\xspace}

\leftskip=\the\parindent
\setlength{\parindent}{0pt}

\titleformat{\section}
  {\normalfont\large\bfseries}
  {}
  {0pt}
  {}
  [{\titlerule[0.8pt]}]

\titlespacing{\section}
  {0pt}
  {1em}
  {\baselineskip}

\newcommand{\comment}[1]{}

\pagestyle{empty}

\title{}
\author{Benjamin Chislett}
\date{}

\begin{document}

\begin{center}
  \Huge\theauthor
  \par
  \large{benjamin.chislett@mail.utoronto.ca}
  \par
  \large\url{https://github.com/benchislett}
\end{center}

\begin{section}{Profile}
Computer Science student with a thirst for knowledge.
Skilled programmer with experience in parallel computing, machine learning, full-stack development, research, and computer graphics.
\\

\textbf{Key Skills:} C++ \vbar Python \vbar High Performance Computing \vbar Deep Learning \vbar Real-Time Ray Tracing

\end{section}

\begin{section}{Experience}

\textbf{Research Intern}
\hfill
\textit{May 2021 - Present}\\*
\textit{EcoSystem Research Lab, University of Toronto}
\begin{itemize}
  \item Actively researching techniques for highly performant machine learning code generation \\
\end{itemize}

\textbf{Machine Learning Engineer}
\hfill
\textit{Apr 2021 - Aug 2021}\\*
\textit{Activeloop AI, California (Remote)}
\begin{itemize}
  \item Designed and developed infrastructure for a cloud machine learning data platform \\
\end{itemize}

\textbf{Software Developer}
\hfill
\textit{Sep 2018 - Jul 2019, May 2020 - Dec 2020}\\*
\textit{Mysa Smart Thermostats, St. John's, NL}
\begin{itemize}
  \item Maintained and rewrote a full stack typescript application used to investigate user accounts distributed over an AWS backend
  \item Authored a suite of libraries for interacting with AWS at multiple tiers of abstraction
  \item Developed various new features for a react-native mobile application
  \item Architected a data pipeline used to create a data lake and perform analytics
  \item Led a small team of developers to create an IoT-based device simulator \\
\end{itemize}

\textbf{Research Intern}
\hfill
\textit{Jul 2019 - Sep 2019}\\*
\textit{Okinawa Institute of Science and Technology, Okinawa Prefecture, Japan}
\begin{itemize}
  \item Researched the Compressive Split-Step Fourier Method for efficiently solving the Gross-Pitaevskii equation
  \item Maintained GPUE: a CUDA/C++ application for simulating Bose-Einstein condensates
  \item Authored GPUE.jl: a JuliaLang-GPU implementation of GPUE
\end{itemize}

\end{section}

\begin{section}{Projects}
\textbf{GPU-Accelerated Ray Tracer}
\begin{itemize}
  \item Read over 100 papers on real-time ray tracing on the GPU
  \item Applied data-oriented design and GPU programming principles to develop an accelerated path tracer in CUDA with support for photorealistic materials and models
\end{itemize}

\end{section}

\begin{section}{Education}
\textbf{Honours Bachelor of Science, Computer Science}
\hfill
\textit{Sep 2019 - Present}\\*
\textit{University of Toronto, Scarborough, ON}
\begin{itemize}
  \item Cumulative GPA: 4.00
  \item Teaching assistant for MATA22 and CSCA48, Winter 2020/21
\end{itemize}

\end{section}

\end{document}

